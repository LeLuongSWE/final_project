%\makeglossaries
\makenoidxglossaries

% Danh mục thuật ngữ và từ viết tắt
\newglossaryentry{API}{
    type=\acronymtype,
    name={API},
    description={Giao diện lập trình ứng dụng (Application Programming Interface)},
    first={API}
}

\newglossaryentry{Caching}{
    type=\acronymtype,
    name={Caching},
    description={Lưu dữ liệu trong bộ nhớ chính để có tốc độ truy xuất nhanh hơn},
    first={API}
}
% \newglossaryentry{EUD}{
%     type=\acronymtype,
%     name={EUD},
%     description={Phát triển ứng dụng người dùng cuối(End-User Development)},
%     first={End-User Development}
% }
% \newglossaryentry{GWT}{
%     type=\acronymtype,
%     name={GWT},
%     description={Công cụ lập trình Javascript bằng Java của Google (Google Web Toolkit)},
%     first={Công cụ lập trình Javascript bằng Java của Google (Google Web Toolkit}
% }
\newglossaryentry{HTML}{
    type=\acronymtype,
    name={HTML},
    description={Ngôn ngữ đánh dấu siêu văn bản (HyperText Markup Language)},
    first={Ngôn ngữ đánh dấu siêu văn bản (HyperText Markup Language)}
}

\newglossaryentry{RESTful API}{
    type=\acronymtype,
    name={RESTful API},
    description={Ngôn ngữ đánh dấu siêu văn bản (HyperText Markup Language)},
    first={Ngôn ngữ đánh dấu siêu văn bản (HyperText Markup Language)}
}

\newglossaryentry{POS}{
    type=\acronymtype,
    name={POS},
    description={Hệ thống bán hàng (Point of Sale)},
    first={Hệ thống bán hàng (Point of Sale)}
}
\newglossaryentry{SPA}{
    type=\acronymtype,
    name={SPA},
    description={Ứng dụng web đơn trang (Single Page Application)},
    first={Ứng dụng web đơn trang (Single Page Application)}
}
% RESTful API & Representational State Transfer Application Programming Interface & Một chuẩn kiến trúc phần mềm dùng để thiết kế các dịch vụ Web (Web Services), giúp các hệ thống khác nhau giao tiếp qua giao thức HTTP. \\
% \hline
% SPA & Single Page Application & Ứng dụng web đơn trang, tải một trang HTML duy nhất và cập nhật nội dung động khi người dùng tương tác mà không cần tải lại toàn bộ trang. \\
% \hline
% POS & Point of Sale & Hệ thống quản lý bán hàng tại quầy, bao gồm cả phần cứng (máy tính, máy in) và phần mềm để xử lý giao dịch thanh toán. \\
% \hline
% DOM & Document Object Model & Mô hình đối tượng tài liệu, một giao diện lập trình ứng dụng (API) cho các tài liệu HTML và XML, biểu diễn cấu trúc trang web dưới dạng cây đối tượng. \\
% \hline
% Virtual DOM & Virtual Document Object Model & Một bản sao nhẹ của DOM thực tế, được React sử dụng để tối ưu hóa hiệu năng bằng cách giảm thiểu số lần thao tác trực tiếp lên DOM thực. \\
% \hline
% Component & Component & Trong React, là các khối xây dựng giao diện người dùng độc lập, có thể tái sử dụng và quản lý trạng thái riêng biệt. \\
% \hline
% Framework & Framework & Bộ khung phần mềm cung cấp các thư viện, công cụ và quy tắc chuẩn để phát triển ứng dụng nhanh chóng và hiệu quả hơn (Ví dụ: Spring Boot). \\
% \hline
% Library & Library & Thư viện mã nguồn, tập hợp các đoạn mã được viết sẵn để thực hiện các tác vụ cụ thể (Ví dụ: ReactJS là thư viện UI). \\
% \hline
% DI & Dependency Injection & Tiêm sự phụ thuộc, một kỹ thuật thiết kế phần mềm giúp giảm sự phụ thuộc chặt chẽ (coupling) giữa các đối tượng bằng cách cung cấp các đối tượng phụ thuộc từ bên ngoài. \\
% \hline
% IoC & Inversion of Control & Đảo ngược điều khiển, một nguyên lý thiết kế trong đó luồng thực thi của chương trình được kiểm soát bởi Framework thay vì bởi mã nguồn của người dùng. \\
% \hline
% ORM & Object-Relational Mapping & Kỹ thuật ánh xạ dữ liệu giữa hệ thống hướng đối tượng (như Java) và cơ sở dữ liệu quan hệ (như PostgreSQL). \\
% \hline
% ACID & Atomicity, Consistency, Isolation, Durability & Bốn tính chất quan trọng của giao dịch cơ sở dữ liệu để đảm bảo tính toàn vẹn dữ liệu: Nguyên tử, Nhất quán, Cô lập, Bền vững. \\
% \hline
% QR Code & Quick Response Code & Mã phản hồi nhanh, một loại mã vạch hai chiều có thể được đọc bởi máy quét hoặc camera điện thoại. \\
% \hline
% RUP & Rational Unified Process & Quy trình phát triển phần mềm hợp nhất, một quy trình công nghệ phần mềm hướng đối tượng chia dự án thành các giai đoạn lặp. \\
% \hline