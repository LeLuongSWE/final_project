\documentclass[../DoAn.tex]{subfiles}
\begin{document}

\section{Đặt vấn đề}
\label{section:1.1}
Trong bối cảnh chuyển đổi số đang diễn ra mạnh mẽ tại Việt Nam, ngành dịch vụ ăn uống (F\&B) là một trong những lĩnh vực tiên phong trong việc áp dụng công nghệ để tối ưu hóa vận hành. Các hệ thống quản lý nhà hàng (POS), đặt món trực tuyến hay quản lý chuỗi cung ứng đã trở nên phổ biến, giúp các doanh nghiệp nâng cao hiệu quả kinh doanh và trải nghiệm khách hàng. Tuy nhiên, phần lớn các giải pháp hiện có trên thị trường thường tập trung vào mô hình nhà hàng gọi món, quán cà phê hoặc chuỗi thức ăn nhanh với thực đơn cố định và quy trình phục vụ tiêu chuẩn.

Ngược lại, mô hình kinh doanh "cơm bình dân" – một nét đặc trưng trong văn hóa ẩm thực và đời sống của người Việt, đặc biệt phục vụ đối tượng nhân viên văn phòng và sinh viên – lại có những đặc thù riêng biệt mà các phần mềm quản lý thông thường chưa đáp ứng tốt. Các đặc thù này bao gồm: áp lực phục vụ cực lớn trong khung giờ cao điểm ngắn (thường chỉ từ 11h30 đến 12h30); thực đơn thay đổi linh hoạt theo ngày dựa trên nguyên liệu chợ đầu mối; và quy trình bán hàng yêu cầu tốc độ xử lý giao dịch tính bằng giây để tránh ùn tắc.

Hiện nay, đa số các quán cơm bình dân vẫn vận hành theo phương thức thủ công hoặc sử dụng các phần mềm POS không chuyên dụng. Việc này dẫn đến nhiều bất cập: Thứ nhất, tốc độ bán hàng bị kìm hãm do nhân viên phải thực hiện nhiều thao tác chọn món phức tạp trên các giao diện không tối ưu, gây ùn tắc cục bộ vào giờ cao điểm. Thứ hai, việc quản lý tồn kho gặp khó khăn do không thể kiểm soát chính xác lượng "suất ăn" đã nấu so với nguyên liệu đầu vào, dẫn đến tình trạng thiếu món cục bộ hoặc dư thừa thức ăn gây lãng phí vào cuối ngày. Thứ ba, chủ kinh doanh thiếu các dữ liệu thống kê chính xác về thói quen ăn uống của khách hàng và xu hướng tiêu thụ từng món theo ngày để đưa ra quyết định nhập hàng tối ưu cho ngày hôm sau.

Việc thiếu vắng một công cụ quản lý chuyên biệt, có khả năng thích ứng với sự thay đổi thực đơn hàng ngày và đảm bảo tốc độ xử lý giao dịch cao, đang là rào cản lớn đối với sự phát triển và chuyên nghiệp hóa của mô hình kinh doanh này. Xuất phát từ nhu cầu thực tế đó, việc nghiên cứu và xây dựng một hệ thống quản lý đặc thù, tập trung vào tốc độ và sự linh hoạt cho mô hình cơm bình dân là một yêu cầu cấp thiết, mang lại giá trị thực tiễn cao cho cả người quản lý, nhân viên vận hành và khách hàng.
\section{Mục tiêu và phạm vi đề tài}
\label{section:1.2}
Trên thị trường hiện nay, các giải pháp quản lý nhà hàng phổ biến thường được thiết kế tối ưu cho mô hình gọi món tiêu chuẩn hoặc chuỗi thức ăn nhanh, dẫn đến tình trạng dư thừa tính năng nhưng lại thiếu đi sự linh hoạt cần thiết khi áp dụng vào đặc thù của quán cơm bình dân. Xuất phát từ thực tế đó, mục tiêu cốt lõi của đề tài là nghiên cứu và xây dựng một hệ thống quản lý chuyên biệt, tập trung giải quyết triệt để bài toán về tốc độ xử lý giao dịch trong khung giờ cao điểm và cơ chế quản lý thực đơn thay đổi theo ngày. Hệ thống hướng tới việc thay thế các phương thức ghi chép thủ công hoặc các phần mềm không chuyên, giúp chủ cửa hàng chuẩn hóa quy trình vận hành và kiểm soát hiệu quả nguồn lực kinh doanh.

Về phạm vi thực hiện, đồ án tập trung phát triển một ứng dụng web hoàn chỉnh bao gồm các phân hệ chức năng gắn liền với quy trình nghiệp vụ thực tế. Cụ thể, đối với nghiệp vụ bán hàng, hệ thống cung cấp giao diện POS tối giản nhằm giảm thiểu thao tác và rút ngắn thời gian thanh toán. Song song với đó, phân hệ quản trị được xây dựng với khả năng thiết lập thực đơn động linh hoạt và cơ chế quản lý tồn kho dựa trên số lượng suất ăn thành phẩm thay vì nguyên liệu thô. Ngoài ra, phạm vi đề tài cũng bao gồm việc xây dựng cổng thông tin cho phép khách hàng tra cứu thực đơn và đặt món trực tuyến, cùng với hệ thống báo cáo thống kê hỗ trợ chủ cửa hàng theo dõi doanh thu và xu hướng tiêu dùng theo thời gian thực.
\section{Định hướng giải pháp}
\label{section:1.3}
Để đảm bảo khả năng truy cập đa nền tảng và dễ dàng triển khai, hệ thống được xây dựng theo kiến trúc Client-Server. Cụ thể, phía máy khách (Client) sẽ được phát triển dưới dạng ứng dụng đơn trang (Single Page Application - SPA) nhằm tối ưu hóa trải nghiệm người dùng, giảm thiểu thời gian tải trang và tăng tốc độ tương tác – yếu tố then chốt cho môi trường phục vụ cường độ cao. Phía máy chủ (Server) sẽ được xây dựng theo mô hình RESTful API, đảm bảo tính tách biệt giữa giao diện và logic xử lý, thuận tiện cho việc mở rộng hoặc tích hợp với các hệ thống khác trong tương lai.Giải pháp cụ thể của đồ án tập trung vào việc xử lý hai bài toán cốt lõi là tốc độ và sự linh hoạt thông qua các kỹ thuật sau. Thứ nhất, đối với quy trình bán hàng, hệ thống sẽ tối ưu hóa giao diện người dùng (UI/UX) và luồng xử lý dữ liệu để giảm thiểu số bước thao tác, đồng thời áp dụng các cơ chế caching (lưu đệm) phía client để tăng tốc độ phản hồi ngay cả khi đường truyền mạng không ổn định. Thứ hai, đối với vấn đề thực đơn động và quản lý tồn kho, hệ thống sẽ thiết kế cơ sở dữ liệu linh hoạt cho phép định nghĩa món ăn theo ngày và xây dựng thuật toán tự động trừ kho dựa trên định lượng suất ăn đã được cấu hình trước, giúp cập nhật trạng thái "hết hàng" theo thời gian thực lên giao diện bán hàng và trang đặt món của khách hàng.Đóng góp chính của đồ án là cung cấp một giải pháp phần mềm quản lý toàn diện và chuyên biệt cho mô hình kinh doanh cơm bình dân, một phân khúc thị trường lớn nhưng chưa được phục vụ thỏa đáng bởi các giải pháp công nghệ hiện có. Kết quả đạt được là một hệ thống hoàn chỉnh có khả năng vận hành thực tế, giúp chủ cửa hàng chuẩn hóa quy trình, kiểm soát chặt chẽ nguồn lực và nâng cao chất lượng phục vụ, đồng thời tạo tiền đề dữ liệu cho các phân tích kinh doanh sâu hơn trong tương lai.
\section{Bố cục đồ án}
\label{section:1.4}
Phần còn lại của báo cáo đồ án tốt nghiệp này được tổ chức thành năm chương với nội dung cụ thể như sau.

Chương 2 tập trung vào việc khảo sát và phân tích yêu cầu của hệ thống. Trong chương này, em sẽ trình bày kết quả khảo sát thực tế về quy trình vận hành tại các quán cơm bình dân, từ đó xác định các tác nhân tham gia và xây dựng biểu đồ Use Case tổng quát cũng như các kịch bản nghiệp vụ chi tiết. Các yêu cầu phi chức năng về hiệu năng, bảo mật và tính khả dụng cũng được phân tích kỹ lưỡng tại đây làm cơ sở cho việc thiết kế hệ thống.

Chương 3 giới thiệu về các công nghệ và nền tảng được sử dụng để phát triển ứng dụng. Nội dung chương sẽ phân tích lý do lựa chọn kiến trúc Single Page Application (SPA) với ReactJS cho phía máy khách và Spring Boot cho phía máy chủ, cùng với hệ quản trị cơ sở dữ liệu PostgreSQL. Các công nghệ bổ trợ như WebSocket cho giao tiếp thời gian thực hay Redis cho cơ chế caching cũng được trình bày nhằm làm rõ tính phù hợp với bài toán đặt ra.

Chương 4 trình bày chi tiết về quá trình thiết kế, hiện thực hóa và đánh giá hệ thống. Nội dung bao gồm các thiết kế kiến trúc tổng thể, thiết kế cơ sở dữ liệu, và thiết kế giao diện người dùng. Phần này cũng mô tả quá trình xây dựng các chức năng cốt lõi như quản lý thực đơn động, xử lý đơn hàng tốc độ cao, và trình bày kết quả kiểm thử hệ thống để chứng minh tính đúng đắn và hiệu quả của giải pháp.

Chương 5 tổng hợp các đóng góp nổi bật và các giải pháp kỹ thuật đặc thù đã áp dụng để giải quyết bài toán nghiệp vụ. Tại đây, em sẽ đi sâu phân tích thuật toán quản lý tồn kho theo suất ăn và cơ chế tối ưu hóa thao tác người dùng trên giao diện bán hàng, những điểm làm nên sự khác biệt của hệ thống so với các sản phẩm đại trà.

Cuối cùng, Chương 6 đưa ra kết luận chung về những kết quả đã đạt được so với mục tiêu ban đầu, đồng thời chỉ ra những hạn chế còn tồn tại và đề xuất các hướng phát triển trong tương lai để hoàn thiện sản phẩm.

\end{document}