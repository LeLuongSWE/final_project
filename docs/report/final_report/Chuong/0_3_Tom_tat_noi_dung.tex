\documentclass[../DoAn.tex]{subfiles}
\begin{document}

\begin{center}
    \Large{\textbf{TÓM TẮT NỘI DUNG ĐỒ ÁN}}\\
\end{center}
\vspace{1cm}

Mô hình kinh doanh quán cơm bình dân, một nét đặc trưng của ẩm thực Việt Nam phục vụ phần lớn sinh viên và người lao động, hiện đang đối mặt với những thách thức lớn trong khâu vận hành và quản lý. Các quy trình thủ công như ghi chép sổ sách hay trí nhớ của nhân viên thường dẫn đến sai sót trong tính toán, chậm trễ trong phục vụ giờ cao điểm và khó khăn trong việc kiểm soát nguyên liệu, gây thất thoát tài chính. Mặc dù thị trường đã có nhiều phần mềm quản lý nhà hàng (POS), nhưng đa số được thiết kế cho mô hình gọi món tiêu chuẩn với quy trình phức tạp, chi phí cao và không đáp ứng được tính linh hoạt đặc thù của quán cơm bình dân như thực đơn thay đổi theo ngày hay tốc độ phục vụ cực nhanh. Nhận thấy khoảng trống này, đồ án tập trung phát triển một hệ thống quản lý chuyên biệt, tối ưu hóa cho quy trình bán hàng nhanh gọn và quản lý linh hoạt.

Cách tiếp cận được lựa chọn là xây dựng một ứng dụng web dựa trên kiến trúc Client-Server, sử dụng ReactJS cho giao diện người dùng (Frontend) và Spring Boot cho phía máy chủ (Backend), kết hợp với cơ sở dữ liệu PostgreSQL. Lựa chọn này đảm bảo khả năng tương thích đa nền tảng, dễ dàng triển khai và mở rộng, đồng thời cung cấp trải nghiệm người dùng mượt mà nhờ công nghệ Single Page Application (SPA).

Giải pháp tổng thể bao gồm việc xây dựng các phân hệ chức năng cốt lõi: một giao diện bán hàng (POS) tối giản cho phép nhân viên tạo đơn và thanh toán chỉ trong vài thao tác chạm; một hệ thống quản lý thực đơn động giúp chủ quán dễ dàng cập nhật món ăn theo ngày; và cơ chế quản lý tồn kho thông minh dựa trên định lượng suất ăn thay vì nguyên liệu thô. Bên cạnh đó, hệ thống còn cung cấp các báo cáo thống kê trực quan về doanh thu và xu hướng tiêu dùng, hỗ trợ chủ quán ra quyết định nhập hàng chính xác.

\begin{flushright}
Sinh viên thực hiện\\
\begin{tabular}{@{}c@{}}
\textit{(Ký và ghi rõ họ tên)}
\end{tabular}
\end{flushright}

\end{document}